% Options for packages loaded elsewhere
\PassOptionsToPackage{unicode}{hyperref}
\PassOptionsToPackage{hyphens}{url}
%
\documentclass[
]{article}
\usepackage{amsmath,amssymb}
\usepackage{lmodern}
\usepackage{iftex}
\ifPDFTeX
  \usepackage[T1]{fontenc}
  \usepackage[utf8]{inputenc}
  \usepackage{textcomp} % provide euro and other symbols
\else % if luatex or xetex
  \usepackage{unicode-math}
  \defaultfontfeatures{Scale=MatchLowercase}
  \defaultfontfeatures[\rmfamily]{Ligatures=TeX,Scale=1}
\fi
% Use upquote if available, for straight quotes in verbatim environments
\IfFileExists{upquote.sty}{\usepackage{upquote}}{}
\IfFileExists{microtype.sty}{% use microtype if available
  \usepackage[]{microtype}
  \UseMicrotypeSet[protrusion]{basicmath} % disable protrusion for tt fonts
}{}
\makeatletter
\@ifundefined{KOMAClassName}{% if non-KOMA class
  \IfFileExists{parskip.sty}{%
    \usepackage{parskip}
  }{% else
    \setlength{\parindent}{0pt}
    \setlength{\parskip}{6pt plus 2pt minus 1pt}}
}{% if KOMA class
  \KOMAoptions{parskip=half}}
\makeatother
\usepackage{xcolor}
\usepackage[margin=1in]{geometry}
\usepackage{color}
\usepackage{fancyvrb}
\newcommand{\VerbBar}{|}
\newcommand{\VERB}{\Verb[commandchars=\\\{\}]}
\DefineVerbatimEnvironment{Highlighting}{Verbatim}{commandchars=\\\{\}}
% Add ',fontsize=\small' for more characters per line
\usepackage{framed}
\definecolor{shadecolor}{RGB}{248,248,248}
\newenvironment{Shaded}{\begin{snugshade}}{\end{snugshade}}
\newcommand{\AlertTok}[1]{\textcolor[rgb]{0.94,0.16,0.16}{#1}}
\newcommand{\AnnotationTok}[1]{\textcolor[rgb]{0.56,0.35,0.01}{\textbf{\textit{#1}}}}
\newcommand{\AttributeTok}[1]{\textcolor[rgb]{0.77,0.63,0.00}{#1}}
\newcommand{\BaseNTok}[1]{\textcolor[rgb]{0.00,0.00,0.81}{#1}}
\newcommand{\BuiltInTok}[1]{#1}
\newcommand{\CharTok}[1]{\textcolor[rgb]{0.31,0.60,0.02}{#1}}
\newcommand{\CommentTok}[1]{\textcolor[rgb]{0.56,0.35,0.01}{\textit{#1}}}
\newcommand{\CommentVarTok}[1]{\textcolor[rgb]{0.56,0.35,0.01}{\textbf{\textit{#1}}}}
\newcommand{\ConstantTok}[1]{\textcolor[rgb]{0.00,0.00,0.00}{#1}}
\newcommand{\ControlFlowTok}[1]{\textcolor[rgb]{0.13,0.29,0.53}{\textbf{#1}}}
\newcommand{\DataTypeTok}[1]{\textcolor[rgb]{0.13,0.29,0.53}{#1}}
\newcommand{\DecValTok}[1]{\textcolor[rgb]{0.00,0.00,0.81}{#1}}
\newcommand{\DocumentationTok}[1]{\textcolor[rgb]{0.56,0.35,0.01}{\textbf{\textit{#1}}}}
\newcommand{\ErrorTok}[1]{\textcolor[rgb]{0.64,0.00,0.00}{\textbf{#1}}}
\newcommand{\ExtensionTok}[1]{#1}
\newcommand{\FloatTok}[1]{\textcolor[rgb]{0.00,0.00,0.81}{#1}}
\newcommand{\FunctionTok}[1]{\textcolor[rgb]{0.00,0.00,0.00}{#1}}
\newcommand{\ImportTok}[1]{#1}
\newcommand{\InformationTok}[1]{\textcolor[rgb]{0.56,0.35,0.01}{\textbf{\textit{#1}}}}
\newcommand{\KeywordTok}[1]{\textcolor[rgb]{0.13,0.29,0.53}{\textbf{#1}}}
\newcommand{\NormalTok}[1]{#1}
\newcommand{\OperatorTok}[1]{\textcolor[rgb]{0.81,0.36,0.00}{\textbf{#1}}}
\newcommand{\OtherTok}[1]{\textcolor[rgb]{0.56,0.35,0.01}{#1}}
\newcommand{\PreprocessorTok}[1]{\textcolor[rgb]{0.56,0.35,0.01}{\textit{#1}}}
\newcommand{\RegionMarkerTok}[1]{#1}
\newcommand{\SpecialCharTok}[1]{\textcolor[rgb]{0.00,0.00,0.00}{#1}}
\newcommand{\SpecialStringTok}[1]{\textcolor[rgb]{0.31,0.60,0.02}{#1}}
\newcommand{\StringTok}[1]{\textcolor[rgb]{0.31,0.60,0.02}{#1}}
\newcommand{\VariableTok}[1]{\textcolor[rgb]{0.00,0.00,0.00}{#1}}
\newcommand{\VerbatimStringTok}[1]{\textcolor[rgb]{0.31,0.60,0.02}{#1}}
\newcommand{\WarningTok}[1]{\textcolor[rgb]{0.56,0.35,0.01}{\textbf{\textit{#1}}}}
\usepackage{graphicx}
\makeatletter
\def\maxwidth{\ifdim\Gin@nat@width>\linewidth\linewidth\else\Gin@nat@width\fi}
\def\maxheight{\ifdim\Gin@nat@height>\textheight\textheight\else\Gin@nat@height\fi}
\makeatother
% Scale images if necessary, so that they will not overflow the page
% margins by default, and it is still possible to overwrite the defaults
% using explicit options in \includegraphics[width, height, ...]{}
\setkeys{Gin}{width=\maxwidth,height=\maxheight,keepaspectratio}
% Set default figure placement to htbp
\makeatletter
\def\fps@figure{htbp}
\makeatother
\setlength{\emergencystretch}{3em} % prevent overfull lines
\providecommand{\tightlist}{%
  \setlength{\itemsep}{0pt}\setlength{\parskip}{0pt}}
\setcounter{secnumdepth}{-\maxdimen} % remove section numbering
\usepackage{booktabs}
\usepackage{longtable}
\usepackage{array}
\usepackage{multirow}
\usepackage{wrapfig}
\usepackage{float}
\usepackage{colortbl}
\usepackage{pdflscape}
\usepackage{tabu}
\usepackage{threeparttable}
\usepackage{threeparttablex}
\usepackage[normalem]{ulem}
\usepackage{makecell}
\usepackage{xcolor}
\ifLuaTeX
  \usepackage{selnolig}  % disable illegal ligatures
\fi
\IfFileExists{bookmark.sty}{\usepackage{bookmark}}{\usepackage{hyperref}}
\IfFileExists{xurl.sty}{\usepackage{xurl}}{} % add URL line breaks if available
\urlstyle{same} % disable monospaced font for URLs
\hypersetup{
  pdftitle={Practica6},
  pdfauthor={David M. Villalobo},
  hidelinks,
  pdfcreator={LaTeX via pandoc}}

\title{Practica6}
\author{David M. Villalobo}
\date{2023-04-09}

\begin{document}
\maketitle

\hypertarget{ej-1}{%
\subsection{Ej 1}\label{ej-1}}

Cargamos las siguientes librerías: MASS, caret, stats, olsrr,
kableExtra, knitrr, rmarkdown.

\begin{Shaded}
\begin{Highlighting}[]
\FunctionTok{library}\NormalTok{(MASS)}
\end{Highlighting}
\end{Shaded}

\begin{verbatim}
## Warning: package 'MASS' was built under R version 4.2.3
\end{verbatim}

\begin{Shaded}
\begin{Highlighting}[]
\FunctionTok{library}\NormalTok{(caret)}
\end{Highlighting}
\end{Shaded}

\begin{verbatim}
## Warning: package 'caret' was built under R version 4.2.3
\end{verbatim}

\begin{verbatim}
## Loading required package: ggplot2
\end{verbatim}

\begin{verbatim}
## Loading required package: lattice
\end{verbatim}

\begin{Shaded}
\begin{Highlighting}[]
\FunctionTok{library}\NormalTok{(stats)}
\FunctionTok{library}\NormalTok{(olsrr)}
\end{Highlighting}
\end{Shaded}

\begin{verbatim}
## Warning: package 'olsrr' was built under R version 4.2.3
\end{verbatim}

\begin{verbatim}
## 
## Attaching package: 'olsrr'
\end{verbatim}

\begin{verbatim}
## The following object is masked from 'package:MASS':
## 
##     cement
\end{verbatim}

\begin{verbatim}
## The following object is masked from 'package:datasets':
## 
##     rivers
\end{verbatim}

\begin{Shaded}
\begin{Highlighting}[]
\FunctionTok{library}\NormalTok{(kableExtra)}
\end{Highlighting}
\end{Shaded}

\begin{verbatim}
## Warning: package 'kableExtra' was built under R version 4.2.3
\end{verbatim}

\begin{verbatim}
## Warning in !is.null(rmarkdown::metadata$output) && rmarkdown::metadata$output
## %in% : 'length(x) = 2 > 1' in coercion to 'logical(1)'
\end{verbatim}

\begin{Shaded}
\begin{Highlighting}[]
\FunctionTok{library}\NormalTok{(knitr)}
\FunctionTok{library}\NormalTok{(rmarkdown)}
\end{Highlighting}
\end{Shaded}

\hypertarget{ej-2}{%
\subsection{Ej 2}\label{ej-2}}

Creamos 2 variables almacenadas como vector: ``y\_cuentas'' y
``x\_distancia''.

\begin{Shaded}
\begin{Highlighting}[]
\NormalTok{y\_cuentas }\OtherTok{\textless{}{-}} \FunctionTok{c}\NormalTok{(}\DecValTok{110}\NormalTok{,}\DecValTok{2}\NormalTok{,}\DecValTok{6}\NormalTok{,}\DecValTok{98}\NormalTok{,}\DecValTok{40}\NormalTok{,}\DecValTok{94}\NormalTok{,}\DecValTok{31}\NormalTok{,}\DecValTok{5}\NormalTok{,}\DecValTok{8}\NormalTok{,}\DecValTok{10}\NormalTok{)}
\NormalTok{x\_distancia }\OtherTok{\textless{}{-}} \FunctionTok{c}\NormalTok{(}\FloatTok{1.1}\NormalTok{,}\FloatTok{100.2}\NormalTok{,}\FloatTok{90.3}\NormalTok{,}\FloatTok{5.4}\NormalTok{,}\FloatTok{57.5}\NormalTok{,}\FloatTok{6.6}\NormalTok{,}\FloatTok{34.7}\NormalTok{,}\FloatTok{65.8}\NormalTok{,}\FloatTok{57.9}\NormalTok{,}\FloatTok{86.1}\NormalTok{)}
\end{Highlighting}
\end{Shaded}

\hypertarget{ej-3}{%
\subsection{Ej 3}\label{ej-3}}

Verificamos el supuesto de linealidad de la variable explicativa
incluyendo un contraste de hipótesis. Con plot() sabremos que presentan
una correlación inversa (cuanto más lejos del 0, más cuentas aparecen).
Con cor.test() vemos la posibilidad de que de que el valor que aparece
como p-value (el cual informa sobre la correlación de las variables) no
se de por azar por completo. Como el p-value es menor de 0.05 podemos
decir que la correlación es fiable.

\begin{Shaded}
\begin{Highlighting}[]
\FunctionTok{plot}\NormalTok{(y\_cuentas, x\_distancia)}
\end{Highlighting}
\end{Shaded}

\includegraphics{Practica6_files/figure-latex/unnamed-chunk-3-1.pdf}

\begin{Shaded}
\begin{Highlighting}[]
\FunctionTok{cor.test}\NormalTok{(y\_cuentas, x\_distancia)}
\end{Highlighting}
\end{Shaded}

\begin{verbatim}
## 
##  Pearson's product-moment correlation
## 
## data:  y_cuentas and x_distancia
## t = -6.8847, df = 8, p-value = 0.0001265
## alternative hypothesis: true correlation is not equal to 0
## 95 percent confidence interval:
##  -0.9824414 -0.7072588
## sample estimates:
##        cor 
## -0.9249824
\end{verbatim}

\hypertarget{ej-4}{%
\subsection{Ej 4}\label{ej-4}}

X es la variable explicativa, no tiene forma de campana, por lo que no
es normal, la regresión no se cumple siempre. Con 10 variables, tenemos
pocos datos, el shapiro.test no es fiable, da una conclusión no valida,
ya que da como resultado que es normal (p-value mayor de 0.05) y eso es
falso.

\begin{Shaded}
\begin{Highlighting}[]
\FunctionTok{hist}\NormalTok{(x\_distancia, }\AttributeTok{probability =} \ConstantTok{TRUE}\NormalTok{)}
\FunctionTok{lines}\NormalTok{(}\FunctionTok{density}\NormalTok{(x\_distancia))}
\end{Highlighting}
\end{Shaded}

\includegraphics{Practica6_files/figure-latex/unnamed-chunk-4-1.pdf}

\begin{Shaded}
\begin{Highlighting}[]
\FunctionTok{shapiro.test}\NormalTok{(x\_distancia)}
\end{Highlighting}
\end{Shaded}

\begin{verbatim}
## 
##  Shapiro-Wilk normality test
## 
## data:  x_distancia
## W = 0.90687, p-value = 0.2602
\end{verbatim}

\hypertarget{ej-5}{%
\subsection{Ej 5}\label{ej-5}}

Multiplicamos las variables con * y lo nombramos xy.

\begin{Shaded}
\begin{Highlighting}[]
\NormalTok{xy }\OtherTok{\textless{}{-}}\NormalTok{ y\_cuentas}\SpecialCharTok{*}\NormalTok{x\_distancia}
\NormalTok{xy}
\end{Highlighting}
\end{Shaded}

\begin{verbatim}
##  [1]  121.0  200.4  541.8  529.2 2300.0  620.4 1075.7  329.0  463.2  861.0
\end{verbatim}

\hypertarget{ej-6}{%
\subsection{Ej 6}\label{ej-6}}

Elevamos al cuadrado la variable x, y le otorgamos el nombre
x\_cuadrado.

\begin{Shaded}
\begin{Highlighting}[]
\NormalTok{x\_cuadrado }\OtherTok{\textless{}{-}}\NormalTok{ x\_distancia}\SpecialCharTok{\^{}}\DecValTok{2}
\NormalTok{x\_cuadrado}
\end{Highlighting}
\end{Shaded}

\begin{verbatim}
##  [1]     1.21 10040.04  8154.09    29.16  3306.25    43.56  1204.09  4329.64
##  [9]  3352.41  7413.21
\end{verbatim}

\hypertarget{ej-7}{%
\subsection{Ej 7}\label{ej-7}}

Creamos un dataframe llamado tabla\_datos con las variables creadas
hasta ahora.

\begin{Shaded}
\begin{Highlighting}[]
\NormalTok{tabla\_datos }\OtherTok{\textless{}{-}} \FunctionTok{data.frame}\NormalTok{(y\_cuentas, x\_distancia, xy, x\_cuadrado)}
\NormalTok{tabla\_datos}
\end{Highlighting}
\end{Shaded}

\begin{verbatim}
##    y_cuentas x_distancia     xy x_cuadrado
## 1        110         1.1  121.0       1.21
## 2          2       100.2  200.4   10040.04
## 3          6        90.3  541.8    8154.09
## 4         98         5.4  529.2      29.16
## 5         40        57.5 2300.0    3306.25
## 6         94         6.6  620.4      43.56
## 7         31        34.7 1075.7    1204.09
## 8          5        65.8  329.0    4329.64
## 9          8        57.9  463.2    3352.41
## 10        10        86.1  861.0    7413.21
\end{verbatim}

\hypertarget{ej-8}{%
\subsection{Ej 8}\label{ej-8}}

Visualizamos el dataframe anteriormente creado como una tabla, gracias a
la librería kableExtra, primero esterilizamos el dataframe y luego la
ejecutamos.

\begin{Shaded}
\begin{Highlighting}[]
\FunctionTok{kbl}\NormalTok{(tabla\_datos)}\SpecialCharTok{\%\textgreater{}\%}
  \FunctionTok{kable\_minimal}\NormalTok{()}
\end{Highlighting}
\end{Shaded}

\begin{table}
\centering
\begin{tabular}[t]{r|r|r|r}
\hline
y\_cuentas & x\_distancia & xy & x\_cuadrado\\
\hline
110 & 1.1 & 121.0 & 1.21\\
\hline
2 & 100.2 & 200.4 & 10040.04\\
\hline
6 & 90.3 & 541.8 & 8154.09\\
\hline
98 & 5.4 & 529.2 & 29.16\\
\hline
40 & 57.5 & 2300.0 & 3306.25\\
\hline
94 & 6.6 & 620.4 & 43.56\\
\hline
31 & 34.7 & 1075.7 & 1204.09\\
\hline
5 & 65.8 & 329.0 & 4329.64\\
\hline
8 & 57.9 & 463.2 & 3352.41\\
\hline
10 & 86.1 & 861.0 & 7413.21\\
\hline
\end{tabular}
\end{table}

\hypertarget{ej-9}{%
\subsection{Ej 9}\label{ej-9}}

Realizamos el sumatorio de los valores almacenados en cada una de las 4
columnas. Para ello utilizamos la función sum más el simbolo \$ para
especificar los valores de que columna queremos sumar.

\begin{Shaded}
\begin{Highlighting}[]
\NormalTok{sum\_y }\OtherTok{\textless{}{-}} \FunctionTok{sum}\NormalTok{(tabla\_datos}\SpecialCharTok{$}\NormalTok{y\_cuentas)}
\NormalTok{sum\_y}
\end{Highlighting}
\end{Shaded}

\begin{verbatim}
## [1] 404
\end{verbatim}

\begin{Shaded}
\begin{Highlighting}[]
\NormalTok{sum\_x }\OtherTok{\textless{}{-}} \FunctionTok{sum}\NormalTok{(tabla\_datos}\SpecialCharTok{$}\NormalTok{x\_distancia)}
\NormalTok{sum\_x}
\end{Highlighting}
\end{Shaded}

\begin{verbatim}
## [1] 505.6
\end{verbatim}

\begin{Shaded}
\begin{Highlighting}[]
\NormalTok{sum\_xy }\OtherTok{\textless{}{-}} \FunctionTok{sum}\NormalTok{(tabla\_datos}\SpecialCharTok{$}\NormalTok{xy)}
\NormalTok{sum\_xy}
\end{Highlighting}
\end{Shaded}

\begin{verbatim}
## [1] 7041.7
\end{verbatim}

\begin{Shaded}
\begin{Highlighting}[]
\NormalTok{sum\_xcuadrado }\OtherTok{\textless{}{-}} \FunctionTok{sum}\NormalTok{(tabla\_datos}\SpecialCharTok{$}\NormalTok{x\_cuadrado)}
\NormalTok{sum\_xcuadrado}
\end{Highlighting}
\end{Shaded}

\begin{verbatim}
## [1] 37873.66
\end{verbatim}

\hypertarget{ej-10}{%
\subsection{Ej 10}\label{ej-10}}

Creamos un vector con los resultados de los sumatorios anteriores, para
añadirlos a un dataframe ya existente con rbind(). Le damos un nombre
diferente al dataframe para crear uno nuevo sin eliminar el original.
Con rownames() indicamos que contiene la nueva fila añadida.

\begin{Shaded}
\begin{Highlighting}[]
\NormalTok{sum\_total }\OtherTok{\textless{}{-}} \FunctionTok{c}\NormalTok{(sum\_y, sum\_x, sum\_xy, sum\_xcuadrado)}
\NormalTok{sum\_total}
\end{Highlighting}
\end{Shaded}

\begin{verbatim}
## [1]   404.00   505.60  7041.70 37873.66
\end{verbatim}

\begin{Shaded}
\begin{Highlighting}[]
\NormalTok{tabla\_datos2 }\OtherTok{\textless{}{-}} \FunctionTok{rbind}\NormalTok{(tabla\_datos, sum\_total)}
\NormalTok{tabla\_datos2}
\end{Highlighting}
\end{Shaded}

\begin{verbatim}
##    y_cuentas x_distancia     xy x_cuadrado
## 1        110         1.1  121.0       1.21
## 2          2       100.2  200.4   10040.04
## 3          6        90.3  541.8    8154.09
## 4         98         5.4  529.2      29.16
## 5         40        57.5 2300.0    3306.25
## 6         94         6.6  620.4      43.56
## 7         31        34.7 1075.7    1204.09
## 8          5        65.8  329.0    4329.64
## 9          8        57.9  463.2    3352.41
## 10        10        86.1  861.0    7413.21
## 11       404       505.6 7041.7   37873.66
\end{verbatim}

\begin{Shaded}
\begin{Highlighting}[]
\FunctionTok{rownames}\NormalTok{(tabla\_datos2)[}\DecValTok{11}\NormalTok{] }\OtherTok{\textless{}{-}} \StringTok{"sumatorio"}
\FunctionTok{rownames}\NormalTok{(tabla\_datos2)[}\DecValTok{11}\NormalTok{]}
\end{Highlighting}
\end{Shaded}

\begin{verbatim}
## [1] "sumatorio"
\end{verbatim}

\begin{Shaded}
\begin{Highlighting}[]
\NormalTok{tabla\_datos2}
\end{Highlighting}
\end{Shaded}

\begin{verbatim}
##           y_cuentas x_distancia     xy x_cuadrado
## 1               110         1.1  121.0       1.21
## 2                 2       100.2  200.4   10040.04
## 3                 6        90.3  541.8    8154.09
## 4                98         5.4  529.2      29.16
## 5                40        57.5 2300.0    3306.25
## 6                94         6.6  620.4      43.56
## 7                31        34.7 1075.7    1204.09
## 8                 5        65.8  329.0    4329.64
## 9                 8        57.9  463.2    3352.41
## 10               10        86.1  861.0    7413.21
## sumatorio       404       505.6 7041.7   37873.66
\end{verbatim}

\hypertarget{ej-11}{%
\subsection{Ej 11}\label{ej-11}}

Utilizamos lm() para calcular los datos que necesitemos para obtener la
recta de regresión. Con summary() obtenemos los datos necesarios para la
recta de regresión. Los datos principales serían 95.37 y -1.087. En la
ecuación quedaría así:

\[B_0 -> y - B_1 · x_1 + Σ_1\] \[Y_0 = 95.37 - 1.087 · x_1 + Σ_1\]

\begin{Shaded}
\begin{Highlighting}[]
\NormalTok{modelo }\OtherTok{\textless{}{-}} \FunctionTok{lm}\NormalTok{(y\_cuentas }\SpecialCharTok{\textasciitilde{}}\NormalTok{ x\_distancia, }\AttributeTok{data =}\NormalTok{ tabla\_datos)}
\FunctionTok{summary}\NormalTok{(modelo)}
\end{Highlighting}
\end{Shaded}

\begin{verbatim}
## 
## Call:
## lm(formula = y_cuentas ~ x_distancia, data = tabla_datos)
## 
## Residuals:
##     Min      1Q  Median      3Q     Max 
## -26.644 -12.672   7.693   8.730  15.825 
## 
## Coefficients:
##             Estimate Std. Error t value Pr(>|t|)    
## (Intercept)  95.3710     9.7188   9.813 9.77e-06 ***
## x_distancia  -1.0872     0.1579  -6.885 0.000126 ***
## ---
## Signif. codes:  0 '***' 0.001 '**' 0.01 '*' 0.05 '.' 0.1 ' ' 1
## 
## Residual standard error: 17.52 on 8 degrees of freedom
## Multiple R-squared:  0.8556, Adjusted R-squared:  0.8375 
## F-statistic:  47.4 on 1 and 8 DF,  p-value: 0.0001265
\end{verbatim}

\hypertarget{ej-12}{%
\subsection{Ej 12}\label{ej-12}}

Creamos una nube de puntos con plot() y añadimos la recta de regresión
con abline().

\begin{Shaded}
\begin{Highlighting}[]
\FunctionTok{plot}\NormalTok{(y\_cuentas }\SpecialCharTok{\textasciitilde{}}\NormalTok{ x\_distancia, }\AttributeTok{main =} \StringTok{"gráfico de dispersión la recta de regresión"}\NormalTok{, }\AttributeTok{ylab =} \StringTok{"Cuentas"}\NormalTok{, }\AttributeTok{xlab =} \StringTok{"Distancia"}\NormalTok{)}
\FunctionTok{abline}\NormalTok{(modelo)}
\end{Highlighting}
\end{Shaded}

\includegraphics{Practica6_files/figure-latex/unnamed-chunk-12-1.pdf}
\#\# Ej 13

Los residuos de un modelo se refieren a las diferencias entre los
valores observados y los valores predichos por el modelo. Los residuos
son una medida de la discrepancia entre los datos observados y los
valores ajustados del modelo y se utilizan para evaluar la calidad del
ajuste del modelo. Si los residuos son grandes, esto puede indicar que
el modelo no se ajusta bien a los datos y que se necesitan ajustes en el
modelo o en los datos. Primero calculamos los residuos estandarizados
con rstandard, después los residuos con resid, y finalmente los
estunderizados con rstudent. Todo utilizando el modelo creado en el
ejercicio 11.

\begin{Shaded}
\begin{Highlighting}[]
\NormalTok{res\_estandarizados }\OtherTok{\textless{}{-}} \FunctionTok{rstandard}\NormalTok{(modelo)}
\NormalTok{res\_estandarizados}
\end{Highlighting}
\end{Shaded}

\begin{verbatim}
##          1          2          3          4          5          6          7 
##  1.0784816  1.0622593  0.5721649  0.5661008  0.4307982  0.3843280 -1.6213545 
##          8          9         10 
## -1.1448722 -1.4726329  0.5266739
\end{verbatim}

\begin{Shaded}
\begin{Highlighting}[]
\NormalTok{residuos }\OtherTok{\textless{}{-}} \FunctionTok{resid}\NormalTok{(modelo)}
\NormalTok{residuos}
\end{Highlighting}
\end{Shaded}

\begin{verbatim}
##          1          2          3          4          5          6          7 
##  15.824925  15.570779   8.807066   8.500073   7.145471   5.804766 -26.643686 
##          8          9         10 
## -18.830406 -24.419631   8.240642
\end{verbatim}

\begin{Shaded}
\begin{Highlighting}[]
\NormalTok{res\_estunderizados }\OtherTok{\textless{}{-}} \FunctionTok{rstudent}\NormalTok{(modelo)}
\NormalTok{res\_estunderizados}
\end{Highlighting}
\end{Shaded}

\begin{verbatim}
##          1          2          3          4          5          6          7 
##  1.0912716  1.0721374  0.5465101  0.5404749  0.4077319  0.3628715 -1.8509341 
##          8          9         10 
## -1.1711614 -1.6134626  0.5014281
\end{verbatim}

\hypertarget{ej-14}{%
\subsection{Ej 14}\label{ej-14}}

Sustituimos el valor de x que queremos calcular en la recta de
regresión.

\begin{Shaded}
\begin{Highlighting}[]
\NormalTok{y\_6\_6 }\OtherTok{\textless{}{-}} \FloatTok{95.37} \SpecialCharTok{{-}} \FloatTok{1.087} \SpecialCharTok{*} \FloatTok{6.6}
\NormalTok{y\_6\_6}
\end{Highlighting}
\end{Shaded}

\begin{verbatim}
## [1] 88.1958
\end{verbatim}

\hypertarget{ej-15}{%
\subsection{Ej 15}\label{ej-15}}

Utilizamos la librería dplyr. Los dos conjuntos de datos creados sirven,
el primero para crear una recta de regresión y el segundo para comprobar
su validez.

\begin{Shaded}
\begin{Highlighting}[]
\FunctionTok{library}\NormalTok{(dplyr)}
\end{Highlighting}
\end{Shaded}

\begin{verbatim}
## 
## Attaching package: 'dplyr'
\end{verbatim}

\begin{verbatim}
## The following object is masked from 'package:kableExtra':
## 
##     group_rows
\end{verbatim}

\begin{verbatim}
## The following object is masked from 'package:MASS':
## 
##     select
\end{verbatim}

\begin{verbatim}
## The following objects are masked from 'package:stats':
## 
##     filter, lag
\end{verbatim}

\begin{verbatim}
## The following objects are masked from 'package:base':
## 
##     intersect, setdiff, setequal, union
\end{verbatim}

\begin{Shaded}
\begin{Highlighting}[]
\NormalTok{data }\OtherTok{\textless{}{-}} \FunctionTok{data.frame}\NormalTok{(x\_distancia, y\_cuentas)}
\NormalTok{train }\OtherTok{\textless{}{-}}\NormalTok{ data }\SpecialCharTok{\%\textgreater{}\%}\NormalTok{ dplyr}\SpecialCharTok{::}\FunctionTok{sample\_frac}\NormalTok{(.}\DecValTok{8}\NormalTok{)}
\NormalTok{test }\OtherTok{\textless{}{-}}\NormalTok{ dplyr}\SpecialCharTok{::}\FunctionTok{anti\_join}\NormalTok{(data, train)}
\end{Highlighting}
\end{Shaded}

\begin{verbatim}
## Joining with `by = join_by(x_distancia, y_cuentas)`
\end{verbatim}

\begin{Shaded}
\begin{Highlighting}[]
\NormalTok{train}
\end{Highlighting}
\end{Shaded}

\begin{verbatim}
##   x_distancia y_cuentas
## 1         6.6        94
## 2        65.8         5
## 3        34.7        31
## 4         1.1       110
## 5        86.1        10
## 6       100.2         2
## 7         5.4        98
## 8        90.3         6
\end{verbatim}

\begin{Shaded}
\begin{Highlighting}[]
\NormalTok{test}
\end{Highlighting}
\end{Shaded}

\begin{verbatim}
##   x_distancia y_cuentas
## 1        57.5        40
## 2        57.9         8
\end{verbatim}

\hypertarget{ej-16}{%
\subsection{Ej 16}\label{ej-16}}

Ajustamos nuevamente el modelo con el conjunto de ``entrenamiento''
creado en el ejercicio anterior, ``train''.

\begin{Shaded}
\begin{Highlighting}[]
\NormalTok{modelo\_2 }\OtherTok{\textless{}{-}} \FunctionTok{lm}\NormalTok{(y\_cuentas }\SpecialCharTok{\textasciitilde{}}\NormalTok{ x\_distancia, train)}
\FunctionTok{summary}\NormalTok{(modelo\_2)}
\end{Highlighting}
\end{Shaded}

\begin{verbatim}
## 
## Call:
## lm(formula = y_cuentas ~ x_distancia, data = train)
## 
## Residuals:
##     Min      1Q  Median      3Q     Max 
## -28.617  -2.153   5.845   8.368  14.294 
## 
## Coefficients:
##             Estimate Std. Error t value Pr(>|t|)    
## (Intercept)  96.8878     9.7183   9.970 5.89e-05 ***
## x_distancia  -1.0741     0.1556  -6.904 0.000457 ***
## ---
## Signif. codes:  0 '***' 0.001 '**' 0.01 '*' 0.05 '.' 0.1 ' ' 1
## 
## Residual standard error: 17.17 on 6 degrees of freedom
## Multiple R-squared:  0.8882, Adjusted R-squared:  0.8695 
## F-statistic: 47.66 on 1 and 6 DF,  p-value: 0.0004565
\end{verbatim}

\hypertarget{ej-17}{%
\subsection{Ej 17}\label{ej-17}}

Los asteríscos nos indican la fiabilidad del modelo. Tres asteríscos
indican una posibilidad baja de que la relación generada entre variables
sea debida al azar.Cuantos menos asteríscos mayor posibilidad de que se
deba al azar. Por otro lado R\^{}2 indica la relación entre los datos y
la línea de regresión. Cuanto más se acerque R\^{}2 a 1 más se acercan
los datos a la línea.

\hypertarget{ej-18}{%
\subsection{Ej 18}\label{ej-18}}

El cálculo para los grados de libertad del modelo se ha obtenido
mediante la función summary().

\hypertarget{ej-19}{%
\subsection{Ej 19}\label{ej-19}}

El total de varianza explicada y no explicada por el modelo se encuentra
con la función anova(), la cual aporta una tabla donde la columna `Sum
Sq' nos proporciona la varianza total explicada y no explicada. Vemos
que la Varianza explicada (10818.7) es baja en comparación a la no
explicada por lo que no podemos pensar que el modelo sea bueno para
realizar predicciones.

\begin{Shaded}
\begin{Highlighting}[]
\FunctionTok{anova}\NormalTok{(modelo\_2)}
\end{Highlighting}
\end{Shaded}

\begin{verbatim}
## Analysis of Variance Table
## 
## Response: y_cuentas
##             Df  Sum Sq Mean Sq F value    Pr(>F)    
## x_distancia  1 14054.6 14054.6   47.66 0.0004565 ***
## Residuals    6  1769.4   294.9                      
## ---
## Signif. codes:  0 '***' 0.001 '**' 0.01 '*' 0.05 '.' 0.1 ' ' 1
\end{verbatim}

\hypertarget{ej-20}{%
\subsection{Ej 20}\label{ej-20}}

Utilizamos el error cuadrático medio. Es una operación de validación
cruzada y ayuda a evaluar el rendimiento de un modelo de regresión
lineal.

\begin{Shaded}
\begin{Highlighting}[]
\NormalTok{Predicci }\OtherTok{\textless{}{-}} \FunctionTok{predict}\NormalTok{(modelo\_2, }\AttributeTok{newdata =}\NormalTok{ test) }
\NormalTok{Predicci}
\end{Highlighting}
\end{Shaded}

\begin{verbatim}
##        1        2 
## 35.12873 34.69910
\end{verbatim}

\begin{Shaded}
\begin{Highlighting}[]
\NormalTok{E\_cadr\_medio }\OtherTok{\textless{}{-}} \FunctionTok{sqrt}\NormalTok{(}\FunctionTok{mean}\NormalTok{((test}\SpecialCharTok{$}\NormalTok{y\_cuentas }\SpecialCharTok{{-}}\NormalTok{ Predicci)}\SpecialCharTok{\^{}}\DecValTok{2}\NormalTok{))}
\NormalTok{E\_cadr\_medio}
\end{Highlighting}
\end{Shaded}

\begin{verbatim}
## [1] 19.19077
\end{verbatim}

\hypertarget{ej-21}{%
\subsection{Ej 21}\label{ej-21}}

Para comprobar si existen valores influyentes en nuestro modelo podemos
usar el método de Cook, que mide la influencia de una observación en el
ajuste del modelo, eliminando la observación y comparando el ajuste del
modelo con y sin ella. Si el ajuste del modelo cambia
significativamente, entonces la observación se considera influyente.
Observamos un valor que descarrila, por lo que podemos considerarlo
influyente.

\begin{Shaded}
\begin{Highlighting}[]
\FunctionTok{plot}\NormalTok{(modelo, }\AttributeTok{which =} \DecValTok{1}\NormalTok{)}
\end{Highlighting}
\end{Shaded}

\includegraphics{Practica6_files/figure-latex/unnamed-chunk-20-1.pdf}

\hypertarget{ej-22}{%
\subsection{Ej 22}\label{ej-22}}

Primero cargamos lmtest y realizamos el test de Durbin-Watson. Hay una
fuerte autocorrelación positiva en los residuos del modelo, ya que el
valor estadístico obtenido es menos de 1, por tanto no se puede asumir
la independencia de los residuos.

\begin{Shaded}
\begin{Highlighting}[]
\FunctionTok{library}\NormalTok{(lmtest)}
\end{Highlighting}
\end{Shaded}

\begin{verbatim}
## Loading required package: zoo
\end{verbatim}

\begin{verbatim}
## 
## Attaching package: 'zoo'
\end{verbatim}

\begin{verbatim}
## The following objects are masked from 'package:base':
## 
##     as.Date, as.Date.numeric
\end{verbatim}

\begin{Shaded}
\begin{Highlighting}[]
\FunctionTok{dwtest}\NormalTok{(modelo, }\AttributeTok{alternative =} \StringTok{"two.sided"}\NormalTok{)}
\end{Highlighting}
\end{Shaded}

\begin{verbatim}
## 
##  Durbin-Watson test
## 
## data:  modelo
## DW = 0.92072, p-value = 0.05819
## alternative hypothesis: true autocorrelation is not 0
\end{verbatim}

\hypertarget{ej-23}{%
\subsection{Ej 23}\label{ej-23}}

El resultado podría indicar que los errores del modelo no son constantes
y que es necesario ajustar el modelo de manera diferente.

\begin{Shaded}
\begin{Highlighting}[]
\FunctionTok{plot}\NormalTok{(}\AttributeTok{x =}\NormalTok{ modelo}\SpecialCharTok{$}\NormalTok{fitted.values, }\AttributeTok{y =}\NormalTok{ modelo}\SpecialCharTok{$}\NormalTok{residuals,}
     \AttributeTok{xlab =} \StringTok{"Valores ajustados"}\NormalTok{, }\AttributeTok{ylab =} \StringTok{"Residuos"}\NormalTok{,}
     \AttributeTok{main =} \StringTok{"Gráfico de residuos versus valores ajustados"}\NormalTok{)}
\FunctionTok{abline}\NormalTok{(}\AttributeTok{h =} \DecValTok{0}\NormalTok{, }\AttributeTok{col =} \StringTok{"red"}\NormalTok{)}
\end{Highlighting}
\end{Shaded}

\includegraphics{Practica6_files/figure-latex/unnamed-chunk-22-1.pdf}

\end{document}
